\documentclass[dvipdfmx,a4paper]{jsarticle}
\usepackage{amsmath,amssymb,amsfonts,amsthm}
\usepackage[dvipdfmx]{graphicx}
\usepackage{tikz}
\usetikzlibrary{positioning, intersections, calc, arrows.meta, math, through, shadows}
\usepackage{tcolorbox}
\tcbuselibrary{theorems,breakable}
\usepackage{enumerate}
\usepackage{mathtools}
\usepackage{otf}
\usepackage{xspace}
\usepackage{newpxtext}
\usepackage[utf8]{inputenc} %中国語コンパイル環境-cjkホットショット
\usepackage{CJKutf8,CJKspace,CJKpunct} %中国語コンパイル環境
\usepackage{pgfplots}
\pgfplotsset{compat=1.18}
\usepackage{okumacro} %漢字ruby
\renewcommand{\abstractname}{注意事項}
\newtagform{textbf}[	extbf]{[}{]}
\usetagform{textbf}
\newcommand*{\ie}{\textbf{\textit{i.e.}}\@\xspace}
\renewcommand{\qedsymbol}{$\blacksquare$}
\newtcbtheorem[]{reidai}{例題}
{fonttitle=\gtfamily\sffamily\bfseries\upshape\large,
colframe=black,colback=black!15!white,
rightrule=1pt,leftrule=1pt,bottomrule=2pt,
colbacktitle=black,theorem style=standard,breakable,arc=10pt}
{tha}
\renewcommand{\thefootnote}{\arabic{footnote}}
\newtheoremstyle{mystyle}%
  {}%                      % 上部スペース
  {}%                      % 下部スペース
  {}%                      % 本文フォント
  {}%                      % 1行目のインデント量
  {\bfseries}%             % 見出しフォント
  :%                       % 見出し後の句読点
  { }%                     % 見出し後のスペース
  {\thmname{#1}\thmnumber{ #2}\thmnote{ (#3)}}
\theoremstyle{mystyle}
% \setcounter{section}{0}
% \stepcounter{section}
% セクションカウンターを使用するが、表示はしない新しいセクションコマンドを作成
\newtheorem{dfn}{\texttt{Def.}}[section]
\newtheorem{exm}[dfn]{\texttt{Ex.}}
\newtheorem{prop}[dfn]{\texttt{Prop.}}
\newtheorem{lem}[dfn]{\texttt{Lem.}}
\newtheorem{thm}[dfn]{\texttt{Thm.}}
\newtheorem{cor}[dfn]{\texttt{Cor.}}
\newtheorem{rem}[dfn]{\texttt{Rem.}}
\newtheorem{fact}[dfn]{\texttt{Fact}}
\renewcommand{\qedsymbol}{$\blacksquare$}
\usepackage{lipsum} % 用于生成示例文本
\usepackage{float} % 强制浮动
\usepackage{tikz} % 用于定位
\usepackage{lastpage}
\usepackage{fancyhdr}
\pagestyle{fancy}
\fancyhf{} 
\fancyhead[L]{\textsc{情報科学Iの第十回講義課題}}
\fancyhead[R]{\textit{www.andreyis.com}}
\fancyfoot[C]{\thepage\quad \textit{of}\quad\pageref*{LastPage}}
\fancypagestyle{plain}{
  \fancyhf{}
  \renewcommand{\headrulewidth}{0pt}
  \fancyfoot[C]{\thepage\quad \textit{of}\quad\pageref*{LastPage}}

}


%排版
\newcommand{\kai}%解答
{\noindent
\begin{tikzpicture}[scale=0.2, baseline=2.8pt]
\draw (3.3,1.2) node{\large\textgt{解 答}};
\draw[thick, rounded corners=3pt,] (0,0)--(6.5,0)--(6.5,2.4)--(0,2.4)--cycle;
\end{tikzpicture}}
\newcommand{\shomei}%証明
{\noindent
\begin{tikzpicture}[scale=0.2, baseline=2.8pt]
\draw (3.3,1.2) node{\textgt{証 明}};
\draw[double,thick,rounded corners=3pt,] (0,0)--(6.5,0)--(6.5,2.4)--(0,2.4)--cycle;
\end{tikzpicture};}
%補足
\newcommand{\hosoku}{\noindent
\begin{tikzpicture}[scale=0.2, baseline=2.8pt]
\draw (6,1) node{\large\textgt{補足}};
\fill (0,1)--(1,0)--(2,1)--(1,2)--cycle;
\fill[gray] (1,1)--(2,0)--(3,1)--(2,2)--cycle;
\fill (2,1)--(3,0)--(4,1)--(3,2)--cycle;
\fill (10,1)--(11,0)--(12,1)--(11,2)--cycle;
\fill[gray] (9,1)--(10,0)--(11,1)--(10,2)--cycle;
\fill (8,1)--(9,0)--(10,1)--(9,2)--cycle;
\end{tikzpicture};}
%注意
\newcommand{\chui}{\noindent
\begin{tikzpicture}[scale=0.2, baseline=2.8pt]
\fill (0,0)--(6.5,0)--(6.5,2.2)--(0,2.2);
\draw (3.3,1) node[white]{\large\textgt{注意!}};
\draw[thick] (0,0)--(6.5,0)--(6.5,2.2)--(0,2.2)--cycle;
\end{tikzpicture};}
\title{\vspace{-3cm} 情報科学Iの第十回講義課題}  %タイトル
\author{\texttt{YI Ran} - $\mathnormal{21122200512}$\\ \texttt{andreyi@outlook.jp}}  %著者名
\date{\today}  %日付
\begin{document}
\maketitle
\thispagestyle{plain}
%\vspace{-0.4cm}
%\begin{figure}[H]
%\centering
%\begin{tikzpicture}[remember picture, overlay]
%   \node[anchor=north east] at (current page.north east) {%
%        \includegraphics[width=2cm]{pics/qr.png} % 修正图片地址
%    };
%    \node[anchor=north east, yshift=-2cm] at (current page.north east) {デジタル版はここ};
%\end{tikzpicture}
%\label{fig:my_label}
%\end{figure}
%\begin{abstract} %概要
  %注意事項
%\end{abstract}
%\begin{reidai}{2次方程式}{解答}
%\end{reidai}
%\begin{proof}
%\end{proof}
\section*{1. 真理値表を用いて、次の等式の成立を示せ}
以下では、真(True)を $1$、偽(False)を $0$ と表記する。

\subsection*{(1) $\neg (p \lor q) = \neg p \land \neg q$}


\begin{table}[h]
\centering
\begin{tabular}{|c|c|c|c|c|c|c|}
\hline
$p$ & $q$ & $p \lor q$ & $\mathbf{\neg (p \lor q)}$ \textbf{(左辺)} & $\neg p$ & $\neg q$ & $\mathbf{\neg p \land \neg q}$ \textbf{(右辺)} \\
\hline
1 & 1 & 1 & \textbf{0} & 0 & 0 & \textbf{0} \\
1 & 0 & 1 & \textbf{0} & 0 & 1 & \textbf{0} \\
0 & 1 & 1 & \textbf{0} & 1 & 0 & \textbf{0} \\
0 & 0 & 0 & \textbf{1} & 1 & 1 & \textbf{1} \\
\hline
\end{tabular}
\end{table}
\noindent
以上より、左辺と右辺の真理値がすべてのパターンで一致するため、等式は成立する。

\subsection*{(2) $\neg (p \land q) = \neg p \lor \neg q$}

\begin{table}[h]
\centering
\begin{tabular}{|c|c|c|c|c|c|c|}
\hline
$p$ & $q$ & $p \land q$ & $\mathbf{\neg (p \land q)}$ \textbf{(左辺)} & $\neg p$ & $\neg q$ & $\mathbf{\neg p \lor \neg q}$ \textbf{(右辺)} \\
\hline
1 & 1 & 1 & \textbf{0} & 0 & 0 & \textbf{0} \\
1 & 0 & 0 & \textbf{1} & 0 & 1 & \textbf{1} \\
0 & 1 & 0 & \textbf{1} & 1 & 0 & \textbf{1} \\
0 & 0 & 0 & \textbf{1} & 1 & 1 & \textbf{1} \\
\hline
\end{tabular}
\end{table}
\noindent
以上より、左辺と右辺の真理値がすべてのパターンで一致するため、等式は成立する。
\newpage
\subsection*{(3) $\neg (p \Rightarrow q) = p \land \neg q$}

\begin{table}[h]
\centering
\begin{tabular}{|c|c|c|c|c|c|}
\hline
$p$ & $q$ & $p \Rightarrow q$ & $\mathbf{\neg (p \Rightarrow q)}$ \textbf{(左辺)} & $\neg q$ & $\mathbf{p \land \neg q}$ \textbf{(右辺)} \\
\hline
1 & 1 & 1 & \textbf{0} & 0 & \textbf{0} \\
1 & 0 & 0 & \textbf{1} & 1 & \textbf{1} \\
0 & 1 & 1 & \textbf{0} & 0 & \textbf{0} \\
0 & 0 & 1 & \textbf{0} & 1 & \textbf{0} \\
\hline
\end{tabular}
\end{table}
\noindent
ここで、 $ \neg (p \Rightarrow q) = \neg (\neg p \lor q)$. よって、左辺と右辺の真理値がすべてのパターンで一致するため、等式は成立する。

\subsection*{(4) $p \lor (q \land r) = (p \lor q) \land (p \lor r)$ (分配律)}

\begin{table}[h]
\centering
\begin{tabular}{|c|c|c|c|c|c|c|c|}
\hline
$p$ & $q$ & $r$ & $q \land r$ & $\mathbf{p \lor (q \land r)}$ \textbf{(左)} & $p \lor q$ & $p \lor r$ & $\mathbf{(p \lor q) \land (p \lor r)}$ \textbf{(右)} \\
\hline
1 & 1 & 1 & 1 & \textbf{1} & 1 & 1 & \textbf{1} \\
1 & 1 & 0 & 0 & \textbf{1} & 1 & 1 & \textbf{1} \\
1 & 0 & 1 & 0 & \textbf{1} & 1 & 1 & \textbf{1} \\
1 & 0 & 0 & 0 & \textbf{1} & 1 & 1 & \textbf{1} \\
0 & 1 & 1 & 1 & \textbf{1} & 1 & 1 & \textbf{1} \\
0 & 1 & 0 & 0 & \textbf{0} & 1 & 0 & \textbf{0} \\
0 & 0 & 1 & 0 & \textbf{0} & 0 & 1 & \textbf{0} \\
0 & 0 & 0 & 0 & \textbf{0} & 0 & 0 & \textbf{0} \\
\hline
\end{tabular}
\end{table}
\noindent
以上より、左辺と右辺の真理値がすべてのパターンで一致するため、等式は成立する。


\section*{2. 論理式の性質を用いて式変形によって等式を示せ}

\subsection*{(1) $\neg (p \Rightarrow \neg q) = p \land q$}

\begin{proof}
含意除去$A \Rightarrow B \equiv \neg A \lor B$ およびド・モルガンの法則を用いて証明する。
\begin{align*}
\text{左辺} &= \neg (p \Rightarrow \neg q) \\
&= \neg (\neg p \lor \neg q) \quad (\text{含意除去}) \\
&= \neg (\neg p) \land \neg (\neg q) \quad (\text{ド・モルガンの法則}) \\
&= p \land q \quad (\text{二重否定}) \\
&= \text{右辺}
\end{align*}
よって、等式は成立する。
\end{proof}

\subsection*{(2) $p \land \neg (\neg p \land q) = p$}

\begin{proof}
ド・モルガンの法則および包含性を用いて証明する。
\begin{align*}
\text{左辺} &= p \land \neg (\neg p \land q) \\
&= p \land (\neg (\neg p) \lor \neg q) \quad (\text{ド・モルガンの法則}) \\
&= p \land (p \lor \neg q) \quad (\text{二重否定}) \\
&= p \quad (\text{包含性 } A \land (A \lor B) = A) \\
&= \text{右辺}
\end{align*}
よって、等式は成立する。
\end{proof}
\section*{3. 次の論理式を主乗法標準形にせよ}

\subsection*{(1) $p \Rightarrow q \land r$}

\kai
\begin{align*}
p \Rightarrow q \land r & = \neg p \lor (q \land r) \quad\\
                        & = (\neg p \lor q) \land (\neg p \lor r) \quad \text{(分配律)}
\end{align*}


したがって、 $\neg p \lor (q \land r)$となる

\subsection*{(2) $\neg(p \land (q \lor r))$}

\kai
\begin{align*}
\neg(p \land (q \lor r)) &= \neg p \lor \neg(q \lor r) \quad \text{(ド・モルガンの法則)} \\
&= \neg p \lor (\neg q \land \neg r) \quad \text{(ド・モルガンの法則)}\\
&= (\neg p \lor \neg q) \land (\neg p \lor \neg r) \quad \text{(分配律)}
\end{align*}

したがって、 $(\neg p \lor \neg q) \land (\neg p \lor \neg r) $となる。

\end{document}