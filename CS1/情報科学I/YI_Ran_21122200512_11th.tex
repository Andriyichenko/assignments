\documentclass[dvipdfmx,a4paper]{jsarticle}
\usepackage{amsmath,amssymb,amsfonts,amsthm}
\usepackage[dvipdfmx]{graphicx}
\usepackage{tikz}
\usetikzlibrary{positioning, intersections, calc, arrows.meta, math, through, shadows}
\usepackage{tcolorbox}
\tcbuselibrary{theorems,breakable}
\usepackage{enumerate}
\usepackage{mathtools}
\usepackage{otf}
\usepackage{xspace}
\usepackage{newpxtext}
\usepackage[utf8]{inputenc} %中国語コンパイル環境-cjkホットショット
\usepackage{CJKutf8,CJKspace,CJKpunct} %中国語コンパイル環境
\usepackage{pgfplots}
\pgfplotsset{compat=1.18}
\usepackage{okumacro} %漢字ruby
\renewcommand{\abstractname}{注意事項}
\newtagform{textbf}[	extbf]{[}{]}
\usetagform{textbf}
\newcommand*{\ie}{\textbf{\textit{i.e.}}\@\xspace}
\renewcommand{\qedsymbol}{$\blacksquare$}
\newtcbtheorem[]{reidai}{例題}
{fonttitle=\gtfamily\sffamily\bfseries\upshape\large,
colframe=black,colback=black!15!white,
rightrule=1pt,leftrule=1pt,bottomrule=2pt,
colbacktitle=black,theorem style=standard,breakable,arc=10pt}
{tha}
\renewcommand{\thefootnote}{\arabic{footnote}}
\newtheoremstyle{mystyle}%
  {}%                      % 上部スペース
  {}%                      % 下部スペース
  {}%                      % 本文フォント
  {}%                      % 1行目のインデント量
  {\bfseries}%             % 見出しフォント
  :%                       % 見出し後の句読点
  { }%                     % 見出し後のスペース
  {\thmname{#1}\thmnumber{ #2}\thmnote{ (#3)}}
\theoremstyle{mystyle}
% \setcounter{section}{0}
% \stepcounter{section}
% セクションカウンターを使用するが、表示はしない新しいセクションコマンドを作成
\newtheorem{dfn}{\texttt{Def.}}[section]
\newtheorem{exm}[dfn]{\texttt{Ex.}}
\newtheorem{prop}[dfn]{\texttt{Prop.}}
\newtheorem{lem}[dfn]{\texttt{Lem.}}
\newtheorem{thm}[dfn]{\texttt{Thm.}}
\newtheorem{cor}[dfn]{\texttt{Cor.}}
\newtheorem{rem}[dfn]{\texttt{Rem.}}
\newtheorem{fact}[dfn]{\texttt{Fact}}
\theoremstyle{definition}
\newtheorem{problem}{問題}
\newtheorem*{solution}{解答}
\renewcommand{\qedsymbol}{$\blacksquare$}
\usepackage{lipsum} % 用于生成示例文本
\usepackage{float} % 强制浮动
\usepackage{tikz} % 用于定位
\usepackage{lastpage}
\usepackage{fancyhdr}
\pagestyle{fancy}
\fancyhf{} 
\fancyhead[L]{\textsc{情報科学Iの第十回講義課題}}
\fancyhead[R]{\textit{www.andreyis.com}}
\fancyfoot[C]{\thepage\quad \textit{of}\quad\pageref*{LastPage}}
\fancypagestyle{plain}{
  \fancyhf{}
  \renewcommand{\headrulewidth}{0pt}
  \fancyfoot[C]{\thepage\quad \textit{of}\quad\pageref*{LastPage}}

}
%排版
\newcommand{\kai}%解答
{\noindent
\begin{tikzpicture}[scale=0.2, baseline=2.8pt]
\draw (3.3,1.2) node{\large\textgt{解 答}};
\draw[thick, rounded corners=3pt,] (0,0)--(6.5,0)--(6.5,2.4)--(0,2.4)--cycle;
\end{tikzpicture}}
\newcommand{\shomei}%証明
{\noindent
\begin{tikzpicture}[scale=0.2, baseline=2.8pt]
\draw (3.3,1.2) node{\textgt{証 明}};
\draw[double,thick,rounded corners=3pt,] (0,0)--(6.5,0)--(6.5,2.4)--(0,2.4)--cycle;
\end{tikzpicture};}
%補足
\newcommand{\hosoku}{\noindent
\begin{tikzpicture}[scale=0.2, baseline=2.8pt]
\draw (6,1) node{\large\textgt{補足}};
\fill (0,1)--(1,0)--(2,1)--(1,2)--cycle;
\fill[gray] (1,1)--(2,0)--(3,1)--(2,2)--cycle;
\fill (2,1)--(3,0)--(4,1)--(3,2)--cycle;
\fill (10,1)--(11,0)--(12,1)--(11,2)--cycle;
\fill[gray] (9,1)--(10,0)--(11,1)--(10,2)--cycle;
\fill (8,1)--(9,0)--(10,1)--(9,2)--cycle;
\end{tikzpicture};}
%注意
\newcommand{\chui}{\noindent
\begin{tikzpicture}[scale=0.2, baseline=2.8pt]
\fill (0,0)--(6.5,0)--(6.5,2.2)--(0,2.2);
\draw (3.3,1) node[white]{\large\textgt{注意!}};
\draw[thick] (0,0)--(6.5,0)--(6.5,2.2)--(0,2.2)--cycle;
\end{tikzpicture};}
\title{\vspace{-3cm} 情報科学Iの第10回講義課題}  %タイトル
\author{\texttt{YI Ran} - $\mathnormal{21122200512}$\\ \texttt{andreyi@outlook.jp}}  %著者名
\date{\today}  %日付
\begin{document}
\maketitle
\thispagestyle{plain}
%\vspace{-0.4cm}
%\begin{figure}[H]
%\centering
%\begin{tikzpicture}[remember picture, overlay]
%   \node[anchor=north east] at (current page.north east) {%
%        \includegraphics[width=2cm]{pics/qr.png} % 修正图片地址
%    };
%    \node[anchor=north east, yshift=-2cm] at (current page.north east) {デジタル版はここ};
%\end{tikzpicture}
%\label{fig:my_label}
%\end{figure}
%\begin{abstract} %概要
  %注意事項
%\end{abstract}
%\begin{reidai}{2次方程式}{解答}
%\end{reidai}
%\begin{proof}
%\end{proof}
\section*{問1. 以下の命題論理の真偽判定}

\subsection*{問題設定}
自然数を定義域とする述語$\mathrm{even}(x)$と$\mathrm{odd}(x)$について、
\begin{itemize}
    \item $\mathrm{even}(x)$: $x$は偶数である
    \item $\mathrm{odd}(x)$: $x$は奇数である
\end{itemize}
とする。次の命題の真偽を判定せよ。

\subsection*{(1) $\neg\mathrm{even}(3)$}

\kai
\begin{align*}
    \mathrm{even}(3)\text{は「3は偶数である」}&\implies 3\text{は奇数なので、}\mathrm{even}(3)\text{は偽}\\
                                      &\implies \neg\mathrm{even}(3)\text{はその否定なので真}
\end{align*}
\noindent
したがって、$\neg\mathrm{even}(3)$は\textbf{真}である。


\subsection*{(2) $\mathrm{even}(24) \land \neg\mathrm{odd}(26)$}
\kai
\vspace{5pt}
\noindent
\begin{align*}
    \mathrm{even}(24)\text{は「24は偶数である」}&\implies \textbf{真}\\
    \mathrm{odd}(26)\text{は「26は奇数である」}&\implies \textbf{偽}\\
    &\implies \neg\mathrm{odd}(26)\textbf{は真}\\
    &\implies \text{真} \land \text{真} = \textbf{真}
\end{align*}
したがって、$\mathrm{even}(24) \land \neg\mathrm{odd}(26)$は\textbf{真}である。


\subsection*{(3) $\neg(\mathrm{even}(24) \land \mathrm{odd}(26))$}
\kai
\vspace{5pt}
\noindent
\begin{align*}
    \mathrm{even}(24)\text{は「24は偶数である」}&\implies \textbf{真}\\
    \mathrm{odd}(26)\text{は「26は奇数である」}&\implies \textbf{偽}\\
    &\implies \mathrm{even}(24) \land \mathrm{odd}(26): \text{真} \land \text{偽} = \textbf{偽}\\
    &\implies \neg(\mathrm{even}(24) \land \mathrm{odd}(26)) = \textbf{真}
\end{align*}
したがって、$\neg(\mathrm{even}(24) \land \mathrm{odd}(26))$は\textbf{真}である。
\section*{問2. 次の命題または術語をgreaterを含む論理式で表せ}

\subsection*{問題設定}
自然数$x, y$について、述語$\mathrm{greater}(x,y)$について、
\begin{itemize}
    \item $\mathrm{greater}(x,y)$: $x$は$y$より大きい
\end{itemize}
とする。次の命題または述語を$\mathrm{greater}$を含む論理式で表せ。

\subsection*{(1) 3は2よりも大きい}
\kai
$$\mathrm{greater}(3, 2)$$

\subsection*{(2) $x$が3よりも大きいならば、$x$は2よりも大きい}

\kai
$$\forall x\, (\mathrm{greater}(x, 3) \rightarrow \mathrm{greater}(x, 2))$$

\subsection*{(3) $x$が$y$よりも大きく、かつ$y$が$z$よりも大きいならば、$x$は$z$よりも大きい}

\kai
$$\forall x\, \forall y\, \forall z\, ((\mathrm{greater}(x, y) \land \mathrm{greater}(y, z)) \rightarrow \mathrm{greater}(x, z))$$


\subsection*{(4) ある$x$が存在して、$x$は100よりも大きい}
\kai
$$\exists x\, \mathrm{greater}(x, 100)$$

\subsection*{(5) すべての$x$について、$x$は100よりも大きいかまたは100よりも大きくない}
\kai
$$\forall x\, (\mathrm{greater}(x, 100) \lor \neg\mathrm{greater}(x, 100))$$


\newpage

\section*{問3. 等式の証明}

\subsection*{問題}
自然数$n$について、次の等式を示せ。
$$1 \cdot 2 + 2 \cdot 3 + \cdots + n \cdot (n+1) = \frac{1}{3}n(n+1)(n+2)$$

\kai
\begin{proof}
数学的帰納法により証明する。

\vspace{10pt}
\noindent
$n = 1$のとき

左辺:
$$1 \cdot 2 = 2$$

右辺:
$$\frac{1}{3} \cdot 1 \cdot 2 \cdot 3 = \frac{6}{3} = 2$$
\noindent
左辺 = 右辺より、$n = 1$のとき成立。

\vspace{10pt}
\noindent
$n = k$のとき成立すると仮定
\noindent
すなわち、
$$1 \cdot 2 + 2 \cdot 3 + \cdots + k(k+1) = \frac{1}{3}k(k+1)(k+2)$$
が成り立つと仮定する。

\vspace{10pt}
\noindent
\textbf{$n = k+1$のとき}
\noindent
左辺を計算する:
\begin{align*}
&1 \cdot 2 + 2 \cdot 3 + \cdots + k(k+1) + (k+1)(k+2)\\
&= \frac{1}{3}k(k+1)(k+2) + (k+1)(k+2) \quad \text{(帰納法の仮定を使用)}\\
&= (k+1)(k+2)\left(\frac{k}{3} + 1\right)\\
&= (k+1)(k+2) \cdot \frac{k+3}{3}\\
&= \frac{1}{3}(k+1)(k+2)(k+3)
\end{align*}

これは$n = k+1$のときの右辺:
$$\frac{1}{3}(k+1)((k+1)+1)((k+1)+2) = \frac{1}{3}(k+1)(k+2)(k+3)$$
と一致する。数学的帰納法により、すべての自然数$n$について等式が成り立つ。
\end{proof}

\newpage

\section*{問4. 不等式の証明}

\subsection*{問題}
すべての実数$x$について、
$$|x+1| - |x-1| \leq 2$$
が成り立つことを示せ。\\

\kai
\begin{proof}
場合分けして考える。

\vspace{10pt}
\noindent
\textbf{【ケース1】}$x \geq 1$のとき

$x + 1 > 0\implies |x+1| = x+1$,\quad  $x - 1 \geq 0\implies |x-1| = x-1$\\
よって、


$$|x+1| - |x-1| = (x+1) - (x-1) = 2$$
\noindent
したがって、$|x+1| - |x-1| = 2 \leq 2$ 

\vspace{10pt}
\noindent
\textbf{【ケース2】}$-1 \leq x < 1$のとき

$x + 1 \geq 0\implies |x+1| = x+1$,\quad $x - 1 < 0\implies |x-1| = -(x-1) = 1-x$\\
よって、

$$|x+1| - |x-1| = (x+1) - (1-x) = 2x$$

$-1 \leq x < 1$より、
$$-2 \leq 2x < 2$$
\noindent
したがって、$|x+1| - |x-1| < 2$ 

\vspace{10pt}
\noindent
\textbf{【ケース3】}$x < -1$のとき

$x + 1 < 0\implies |x+1| = -(x+1) = -x-1$,\quad $x - 1 < 0\implies |x-1| = -(x-1) = 1-x$\\
よって、

$$|x+1| - |x-1| = (-x-1) - (1-x) = -2$$
\noindent
したがって、$|x+1| - |x-1| = -2 \leq 2$ 

\vspace{10pt}
\noindent
\textbf{以上より、すべての場合において$|x+1| - |x-1| \leq 2$が成り立つ。}
\end{proof}
\end{document}