\documentclass[dvipdfmx,a4paper]{jsarticle} % 日文 jsarticle 文类,A4 纸,dvipdfmx 驱动(适合日文+图像)
\usepackage{amsmath,amssymb,amsfonts,amsthm} % AMS 数学环境与符号(对齐公式、定理等)
\usepackage[dvipdfmx]{graphicx}              % 插图:\includegraphics
\usepackage{tikz}                            % TikZ 绘图
\usetikzlibrary{positioning, intersections, calc, arrows.meta, math, through, shadows,automata} % TikZ 扩展库
\usepackage{caption}                        % 图表标题控制
\usepackage{tcolorbox}                       % 彩色盒子环境(例题框、定理框等)
\tcbuselibrary{theorems,breakable}           % tcolorbox 的 “定理环境 + 可分页” 支持
\usepackage{enumerate}                       % 自定义编号列表,例如 \begin{enumerate}[(1)]
\usepackage{mathtools}                       % 对 amsmath 的扩展(:=号等)
\usepackage{extarrows}                       % 更多数学箭头
\usepackage{otf}                             % 和文 OTF 字体支持
\usepackage{xspace}                          % 自动在命令后加适当空格(给 \ie 之类命令用)
\usepackage{newpxtext}                       % Palatino 风格西文字体(正文)
\usepackage[utf8]{inputenc} %中国語コンパイル環境-cjkホットショット % 传统 pLaTeX 下的 UTF-8 输入编码
\usepackage{CJKutf8,CJKspace,CJKpunct} %中国語コンパイル環境 % CJK 中文/日文环境(配合 CJK 环境使用)
\usepackage{pgfplots}                        % 函数图、数据图绘制(基于 TikZ)
\usepackage{lastpage}                        % 获取最后一页页码(用来做 “Page x of y”)
\usepackage{fancyhdr}                        % 自定义页眉页脚
\pagestyle{fancy}                            % 使用 fancyhdr 作为默认页式
\fancyhf{}                                   % 清空默认页眉页脚
\fancyhead[L]{\textsc{情報科学Iの第十四回講義課題}} % 左页眉:课程/报告名称(自行修改)
\fancyhead[R]{\textit{www.andreyis.com}}    % 右页眉:网站/个人信息(自行修改)
\fancyfoot[C]{\thepage\quad \textit{of}\quad\pageref*{LastPage}} % 页脚中:Page x of y

\fancypagestyle{plain}{                      % 定义 plain 页式(章节首页等)
  \fancyhf{}                                 %   清空页眉页脚
  \renewcommand{\headrulewidth}{0pt}         %   去掉页眉横线
  \fancyfoot[C]{\thepage\quad \textit{of}\quad\pageref*{LastPage}} % 仍然保留页脚 Page x of y
}

\pgfplotsset{compat=1.18}                    % pgfplots 的版本兼容设置
\usepackage{okumacro} %漢字ruby              % 日文汉字 ruby(\ruby{漢字}{かんじ})

\renewcommand{\abstractname}{注意事項}       % 把 abstract 环境的标题 “Abstract” 改成 “注意事項”

\newtagform{textbf}[\textbf]{[}{]}           % 定义新的公式标签样式:加粗的 [1]
\usetagform{textbf}                          % 使用上面定义的标签样式

\newcommand*{\ie}{\textbf{\textit{i.e.}}\@\xspace} % 定义 \ie 命令(粗斜体 “i.e.”,自动空格)

\renewcommand{\qedsymbol}{$\blacksquare$}   % 证明结尾的 QED 符号改为黑方块 ■(amsthm 的 proof 环境用)

\newtcbtheorem[]{reidai}{例題}               % 定义 tcolorbox 风格的 “例題” 定理环境:\begin{reidai}{标题}{副标题}
{fonttitle=\gtfamily\sffamily\bfseries\upshape\large, % 标题字体:日文ゴシック+无衬线+粗体
 colframe=black,colback=black!15!white,     % 边框黑色,背景浅灰
 rightrule=1pt,leftrule=1pt,bottomrule=2pt, % 右/左/下边框线条粗细
 colbacktitle=black,theorem style=standard,breakable,arc=10pt} % 标题背景黑,正文可分页,圆角
{tha}                                       % 该定理环境的内部 name(用于交叉引用)

\renewcommand{\thefootnote}{\arabic{footnote}} % 脚注编号用阿拉伯数字

\newtheoremstyle{mystyle}%                  % 定义新的定理样式 mystyle
  {}%                      % 上部スペース (定理环境前的空白)
  {}%                      % 下部スペース (定理环境后的空白)
  {}%                      % 本文フォント (正文字体,默认)
  {}%                      % 1行目のインデント量 (标题行缩进)
  {\bfseries}%             % 見出しフォント (标题字体为粗体)
  :%                       % 見出し後の句読点 (标题后面的标点,这里是冒号)
  { }%                     % 見出し後のスペース (标题后与正文之间的空格)
  {\thmname{#1}\thmnumber{ #2}\thmnote{ (#3)}} % 标题格式:Thm 1 (备注)

\theoremstyle{mystyle}                       % 使用刚才定义的 mystyle 作为当前定理样式

% \setcounter{section}{0}                    % (示例)设置 section 计数器为 0(目前注释掉)
% \stepcounter{section}                      % (示例)把 section 计数器加一(目前注释掉)
% セクションカウンターを使用するが、表示はしない新しいセクションコマンドを作成 % 说明文字

\newtheorem{dfn}{\texttt{Def.}}[section]    % “定义”环境 dfn,标题显示为 “Def.”,按 section 编号
\newtheorem{exm}[dfn]{\texttt{Ex.}}         % “例子”环境 exm,与 dfn 共用同一个编号
\newtheorem{prop}[dfn]{\texttt{Prop.}}      % 命题环境 prop
\newtheorem{lem}[dfn]{\texttt{Lem.}}        % 引理环境 lem
\newtheorem{thm}[dfn]{\texttt{Thm.}}        % 定理环境 thm
\newtheorem{cor}[dfn]{\texttt{Cor.}}        % 推论环境 cor
\newtheorem{rem}[dfn]{\texttt{Rem.}}        % 备注环境 rem
\newtheorem{fact}[dfn]{\texttt{Fact}}       % 事实环境 fact

\renewcommand{\qedsymbol}{$\blacksquare$}   % 再次确认 QED 符号为黑方块(如果前面被别的包改掉)

\usepackage{lipsum} % 用于生成示例文本(\lipsum[1-2] 生成假文)
\usepackage{float} % 强制浮动(使用 [H] 选项固定图表位置)
\usepackage{tikz} % 用于定位(重复引入 TikZ,其实可以和前面的合并)

% 排版相关的自定义命令:印 “解答 / 証明 / 補足 / 注意” 标签的小盒子

\newcommand{\kai}% 解答框标题:“解答”
{\noindent
\begin{tikzpicture}[scale=0.2, baseline=2.8pt]
\draw (3.3,1.2) node{\large\textgt{解 答}}; % 文字“解答”
\draw[thick, rounded corners=3pt,] (0,0)--(6.5,0)--(6.5,2.4)--(0,2.4)--cycle; % 带圆角的矩形外框
\end{tikzpicture}} % 使用方式:写在解答前一行,\kai

\newcommand{\shomei}% 証明框标题:“証明”
{\noindent
\begin{tikzpicture}[scale=0.2, baseline=2.8pt]
\draw (3.3,1.2) node{\textgt{証 明}}; % 文字“証明”
\draw[double,thick,rounded corners=3pt,] (0,0)--(6.5,0)--(6.5,2.4)--(0,2.4)--cycle; % 双线矩形外框
\end{tikzpicture};} % 使用方式:写在证明前一行,\shomei

% 補足框
\newcommand{\hosoku}{\noindent
\begin{tikzpicture}[scale=0.2, baseline=2.8pt]
\draw (6,1) node{\large\textgt{補足}}; % 文字“補足”
% 左侧小菱形装饰
\fill (0,1)--(1,0)--(2,1)--(1,2)--cycle;
\fill[gray] (1,1)--(2,0)--(3,1)--(2,2)--cycle;
\fill (2,1)--(3,0)--(4,1)--(3,2)--cycle;
% 右侧小菱形装饰
\fill (10,1)--(11,0)--(12,1)--(11,2)--cycle;
\fill[gray] (9,1)--(10,0)--(11,1)--(10,2)--cycle;
\fill (8,1)--(9,0)--(10,1)--(9,2)--cycle;
\end{tikzpicture};} % 使用方式:\hosoku 后面接补充说明文字

% 注意框
\newcommand{\chui}{\noindent
\begin{tikzpicture}[scale=0.2, baseline=2.8pt]
\fill (0,0)--(6.5,0)--(6.5,2.2)--(0,2.2); % 填满背景色(默认黑)
\draw (3.3,1) node[white]{\large\textgt{注意!}}; % 中间白字“注意!”
\draw[thick] (0,0)--(6.5,0)--(6.5,2.2)--(0,2.2)--cycle; % 外框
\end{tikzpicture};} % 使用方式:\chui 后面写注意内容

% 标题信息:可以根据实际报告修改 title / author / date

\title{\vspace{-3cm} 情報科学Iの第十四回講義課題}  % 标题:目前设置为“空”,仅挤掉竖直空间(可改成真正的标题)
\author{\texttt{YI Ran} - $\mathnormal{21122200512}$\\ \texttt{andreyi@outlook.jp}}  % 作者:姓名+学号+邮箱
\date{\today}  % 日期:自动使用当天日期

\begin{document}
\maketitle                                   % 输出标题页
\thispagestyle{plain}                        % 这一页使用 plain 样式(我们前面定义过)

% 下方是可选的 QR 码 + 文字示例,目前全部注释掉
% \vspace{-0.4cm}
% \begin{figure}[H]
% \centering
% \begin{tikzpicture}[remember picture, overlay]
%    \node[anchor=north east] at (current page.north east) {%
%         \includegraphics[width=2cm]{pics/qr.png} % 这里改成你真正的 QR 图片路径
%     };
%     \node[anchor=north east, yshift=-2cm] at (current page.north east) {デジタル版はここ};
% \end{tikzpicture}
% \label{fig:my_label}
% \end{figure}

% \begin{abstract} %概要
  % 注意事項:如果需要写“注意事项”,可取消注释此 abstract 环境
% \end{abstract}

% 例題环境示例(使用 \begin{reidai}{主标题}{副标题})
% \begin{reidai}{2次方程式}{解答}
% ここに例題の内容を書く。
% \end{reidai}
% \begin{proof}
% ここに証明を書く。
% \end{proof}

\section*{\textbf{\large 問1}\quad \normalfont\large ユークリッドの互除法によって、323と187の最大公約数を求めよ} % 第一节标题(可以直接改文字)
\kai\\


\begin{align*}
323 &= 187 \times 1 + 136\\
\implies 187 &= 136 \times 1 + 51\\
\implies 136 &= 51 \times 2 + 34\\
\implies 51 &= 34 \times 1 + 17\\
\implies 34 &= 17 \times 2 + 0
\end{align*}
よって、323と187の最大公約数は17である。
\vspace{1em}
\section*{\textbf{\large 問2}\quad \normalfont\large $\dfrac{77}{57}$を連分数に展開せよ }
\kai\\

\begin{align*}
  \dfrac{77}{57} &= 1 + \dfrac{20}{57} \\
   \implies\dfrac{57}{20}&= 2 + \dfrac{17}{20}\\
    \implies\dfrac{20}{17}&= 1 + \dfrac{3}{17} \\
    \implies\dfrac{17}{3}&= 5 + \dfrac{2}{3}\\ 
    \implies\dfrac{3}{2}&= 1 + \dfrac{1}{2}\\ 
\end{align*}
したがって、$\dfrac{77}{57} = 1 + \dfrac{1}{2 + \dfrac{1}{1 + \dfrac{1}{5 + \dfrac{1}{1 + \dfrac{1}{2}}}}}$
\newpage
\section*{\textbf{\large 問3}\quad \normalfont\large $111x+30y=12$を満たす整数$x$, $y$を$1$組求めよ }
\kai\\

\noindent
$111, 30, 12$はすべて3の倍数であるため、両辺を3で割ると、
$$
  37x + 10y = 4 \quad\quad\cdots (1)
$$
となる。$(1)$式によって、x、yを求める。\\
まず、$37x + 10y = 1$を求めると、
\begin{align*}
  37 &= 10\times 3 + 7\quad\quad\cdots (a) \\
  \implies 10 &= 7\times 1 + 3 \quad\quad\cdots (b)\\
  \implies 7  &= 3\times 2 + 1 \quad\quad\cdots (c)
\end{align*}
$(c)$式によって、$1 = 7 -3\times 2$となり、$(b)$式によって、$3 = 10 - 7\times 1$となり、$(a)$式によって、$7 = 37 - 10\times 3$となる。
次に、$(a)$と$(b)$を$(c)$に代入すると、
\begin{align*}
  1 &= 7 - 2\times (10 - 7\times 1) \quad\quad\cdots \text{(b)}を代入\\
    &= 3\times 7 - 2\times 10 \\
    &= 3\times (37 - 10\times 3) - 2\times 10 \quad\quad\cdots \text{(a)}を代入\\
    &= 3\times 37 + 10\times (-11)
\end{align*}
よって、$x = 3, y = -11$である。\\
次に、$(1)$式を用いて、式の両辺を4倍すると、$x = 3\times 4 = 12, y = -11\times 4 = -44$である。\\
したがって、$111x + 30y = 12$を満たす整数$x, y$の1組は、$x = 12, y = -44$である。
\vspace{1em}

\section*{\textbf{\large 問4}\quad \normalfont\large $9409x+9991y=97$を満たす整数$x$と整数$y$の関係を示せ }
\kai\\

まず、$9409x + 9991y = 97$を満たす整数$x, y$を求める。\\
$9409, 9991, 97$はすべて97の倍数(\ie $\gcd(9409,9991) = 1$)であるため、両辺を97で割ると、
$$
  97x + 103y = 1 \quad\quad\cdots (1)
$$
となる。すなわち、$(1)$式によって、x、yを求めればよい。\\
まず、$97x + 103y = 1$を求めると、
\begin{align*}
  103 &= 97\times 1 + 6\quad\quad\cdots (a) \\
  \implies 97 &= 6\times 16 + 1 \quad\quad\cdots (b)\\
\end{align*}
$(b)$式によって、$1 = 97 - 6\times 16$となり、$(a)$式によって、$6 = 103 - 97\times 1$となる。
次に、$(a)$を$(b)$に代入すると、
\begin{align*}
  1 &= 97 -  (103 - 97\times 1)\times 16 \\
    &= 17\times 97 + 103\times (-16)
\end{align*}
よって、$x = 17, y = -16$である。\\
したがって、$9409x + 9991y = 97$を満たす整数$x, y$の1組は、$x = 17, y = -16$である。$(1)$に代入すると、\\
$$
  97 \times 17 + 103 \times (-16) = 1\quad\quad\cdots (2)
$$
となる。$(1) - (2)$すると、\\
$$
  97(x - 17) + 103(y + 16) = 0
$$
となる。すなわち、$97(x - 17) = -103(y + 16)$である。\\
ここで、$\gcd(97, 103) = 1$であるため、$\dfrac{x-17}{103} = \dfrac{-(y + 16)}{97}$となる。\\
$97$と$103$は互いに素であるより、\\
$9409x + 9991y = 97$を満たす整数$x, y$は $\begin{cases}x - 17 = 103t\\ y + 16 = -97t \end{cases} (t \in \mathbb{Z})\implies\begin{cases} x = 17 + 103t \\ y = -16 - 97t \end{cases} (t \in \mathbb{Z})$と表せる。\\

\vspace{1em}
\section*{\textbf{\large 問5}\quad \normalfont\large 法17における次の逆数を求めよ。(1) 2,\ (2) 3,\ (3) 4,\ (4) 5 }
\kai\\

\noindent
\textbf{(1)}\quad 2の逆数\\
$2b \equiv 1\quad (\textrm{mod}\ 17)$を満たす整数$b$を求める。ここで、$0\leq b \leq 16$である。\\
$2\times 9 = 17\times 1 + 1$である。\\
したがって、$2$の逆数は$9$である。\\ 
\vspace{1em}

\noindent
\textbf{(2)}\quad 3の逆数\\
$3b \equiv 1\quad (\textrm{mod}\ 17)$を満たす整数$b$を求める。ここで、$0\leq b \leq 16$である。\\
$3\times 6 = 17\times 1 + 1$ である。\\    
したがって、$3$の逆数は$6$である。\\ 
\vspace{1em}

\noindent
\textbf{(3)}\quad 4の逆数\\
$4b \equiv 1\quad (\textrm{mod}\ 17)$を満たす整数$b$を求める。ここで、$0\leq b \leq 16$である。\\
$4\times 13 = 17\times 3 + 1$である。したがって、$4$の逆数は$13$である。\\ 
\vspace{1em}

\noindent
\textbf{(4)}\quad 5の逆数\\
$5b \equiv 1\quad (\textrm{mod}\ 17)$を満たす整数$b$を求める。ここで、$0\leq b \leq 16$である。\\
$5\times 7 = 17\times 2 + 1$である。\\
したがって、$5$の逆数は$7$である。

\vspace{1em}
\section*{\textbf{\large 問6}\quad \normalfont\large 次の合同方程式を満たす最小の正の整数xを求めよ}
\noindent
\textbf{(1)} $7x\equiv 3 \pmod{5}$\\
\textbf{(2)} $5x\equiv 15 \pmod{13}$\\
\textbf{(3)} $17x\equiv 3 \pmod{29}$\\
\textbf{(4)} $x\equiv 1 \pmod{3}, x\equiv 3 \pmod{7}, x\equiv 5 \pmod{11}$ (連立式)\\

\noindent
\kai\\

\noindent
\textbf{(1)}\quad $7x\equiv 3 \pmod{5}$\\
$7 \equiv 2\pmod{5}$であるため、$2x \equiv 3\pmod{5}$となる。\\
$2\times 4 = 5\times 1 + 3$である。\\
したがって、最小の正の整数$x$は$4$である。\\

\noindent
\textbf{(2)}\quad  $5x\equiv 15 \pmod{13}$\\
$15 \equiv 2\pmod{13}$であるため、$5x \equiv 2\pmod{13}$となる。\\
$5\times 3 = 13\times 1 + 2$である。\\
したがって、最小の正の整数$x$は$3$である。\\

\noindent
\textbf{(3)}\quad  $17x\equiv 3 \pmod{29}$\\
$29$は素数であり、$17 < 29$ かつ $\gcd(17, 29) = 1$であるため、17の逆数を求める。\\
\begin{align*}
  29 &= 17\times 1 + 12\quad\quad\cdots (a) \\
  \implies 17 &= 12\times 1 + 5 \quad\quad\cdots (b)\\
  \implies 12 &= 5\times 2 + 2 \quad\quad\cdots (c)\\
  \implies 5  &= 2\times 2 + 1 \quad\quad\cdots (d)\\
\end{align*}
$(d)$式によって、$1 = 5 - 2\times 2$となり、$(c)$式によって、$2 = 12 - 5\times 2$となり、$(b)$式によって、$5 = 17 - 12\times 1$となり、$(a)$式によって、$12 = 29 - 17\times 1$となる。\\
次に、$(a)$と$(b)$と$(a)$を$(d)$に代入すると、
\begin{align*}
  1 &= 5 - 2\times (12 - 5\times 2) \quad\quad\cdots (c)を代入\\
    &= 5 - 2\times 12 + 5\times 4\\
    &= 5\times 5 - 2\times 12 \\
    &= 5\times (17 - 12\times 1) - 2\times 12 \quad\quad\cdots (b)を代入\\
    &= 5\times 17 + 12\times (-7) \\
    &= 5\times 17 + (29 - 17\times 1)\times (-7) \quad\quad\cdots (a)を代入\\
    &= 12\times 17 + 29\times (-7)
\end{align*}

\noindent
よって、$17$の逆数は$12$である。\\
次に、$17x \equiv 3\pmod{29}$に$17$の逆数$12$をかけると、$x \equiv 36 \pmod{29}$となる。\\
$36 \equiv 7\pmod{29}$であるため、最小の正の整数$x$は$7$である。\\

\noindent
\textbf{(4)}\quad  $x\equiv 1 \pmod{3}, x\equiv 3 \pmod{7}, x\equiv 5 \pmod{11}$\\
まず、$N = 3 \times 7 \times 11 = 231$とする。\\
次に、$N_1 = \dfrac{N}{3} = 77, N_2 = \dfrac{N}{7} = 33, N_3 = \dfrac{N}{11} = 21$とする。\\
次に、$N_1, N_2, N_3$の逆数を求める。\\
$77\times u_1 \equiv 1\pmod{3} \rightarrow u_1 =2$\\
$33\times u_2 \equiv 1\pmod{7} \rightarrow u_2 =3$\\
$21\times u_3 \equiv 1\pmod{11} \rightarrow u_3 =10$\\
したがって、$x$は次のように表せる。\\
$$
  x \equiv a_1N_1u_1 + a_2N_2u_2 + a_3N_3u_3 \pmod{N}
$$
ここで、$a_1 = 1, a_2 = 3, a_3 = 5$であるため、\\
\begin{align*}
  x &\equiv 1\times 77 \times 2 + 3\times 33 \times 3 + 5\times 21 \times 10 \pmod{231} \\
    &\equiv 154 + 297 + 1050 \pmod{231} \\
    &\equiv 1501 \pmod{231} \\
    &\equiv 115 \pmod{231}
\end{align*}
よって、最小の正の整数$x$は$115$である。\\ 

\end{document}