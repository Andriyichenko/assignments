% !!! 编译说明: 请使用 XeLaTeX 编译器

\documentclass[12pt]{article}
\usepackage{amssymb}
\usepackage{url}
\usepackage[hidelinks]{hyperref}
% -----------------------------------------------------------------
% 1. 页面设置
% -----------------------------------------------------------------
\usepackage[a4paper, margin=2.5cm]{geometry}

% -----------------------------------------------------------------
% 2. 行距设置
% -----------------------------------------------------------------
\usepackage{setspace}
\doublespacing

% -----------------------------------------------------------------
% 3. 字体设置
% -----------------------------------------------------------------
\usepackage{fontspec}
\setmainfont{Times New Roman}

% -----------------------------------------------------------------
% 4. 段落格式
% -----------------------------------------------------------------
\usepackage{parskip}
\setlength{\parindent}{1.27cm} % APA标准首行缩进
\setlength{\parskip}{0pt}      % 段间距设为0 (由行距控制)

% -----------------------------------------------------------------
% 5. 页眉设置
% -----------------------------------------------------------------
\usepackage{fancyhdr}
\pagestyle{fancy}
\fancyhf{} 
\rhead{\thepage} 
\renewcommand{\headrulewidth}{0pt}

% -----------------------------------------------------------------
% 6. 个人信息录入
% -----------------------------------------------------------------
\newcommand{\myTitle}{Reflections on Space, Dimensions and Language: \\Analyzing ``The Window of the Spaceship `In-Between'\,''} 
\newcommand{\myAuthor}{Yi Ran -- 21122200512}
\newcommand{\myAffiliation}{Department of Mathematical Sciences, Ritsumeikan University} 
\newcommand{\myCourse}{35091: Cross-cultural Seminar (G1)}
\newcommand{\myInstructor}{Dr. Shoji}
\newcommand{\myDate}{\today}

% -----------------------------------------------------------------
% 正文开始
% -----------------------------------------------------------------
\begin{document}

% =================================================================
% 封面页
% =================================================================
\begin{center}
    \vspace*{3cm} 
    
    \textbf{\Large \myTitle} 
    
    \vspace{2\baselineskip} 
    
    \myAuthor \\
    \myAffiliation \\
    \myCourse \\
    \myInstructor \\
    \myDate
\end{center}

\newpage 
% =================================================================

% --- 正文部分 ---

% 正文第一页顶部重复标题
\begin{center}
    \textbf{\myTitle}
\end{center}
\vspace{\baselineskip} 

The concept of communication is often taken for granted in our daily lives. We assume that if we speak the same language, we understand each other. However, The play \textit{The Window of the Spaceship ``In-Between''} challenges this assumption effectively. Through the reading of the script, the viewing of the performance, and the workshop with the actor Leon Koh Yonekawa, I have come to realize that language is not just a tool for information transfer, but a complex spatial phenomenon. This essay reflects on these experiences, focusing on the interplay between theatrical space and mathematical dimensions.

This connects deeply with Toshiki Okada’s column “It’s tough to handle the language” for Kyoto Experiment in 2024. In this text, he explains that the spaceship in the play has a cultural mission: the crew must teach a language to intelligent aliens so that this language will survive and continue in the future. He also writes that all four crew members are played by actors whose first language is not Japanese, because he wants Japanese spoken by non-native speakers to become normal on stage and to break the usual patterns of theatrical dialogue. Reading this column helped me understand that the “cold” attitude of the crew toward Yoshinogari is not only a moral problem inside the story, but also a reflection of how real societies often value “proper” native speech more than the effort of those who speak with an accent.

Okada also describes how the same production was received differently in Tokyo, Kyoto, China, Brussels, and Seoul. In Japan and China, he felt that the central idea of non-native Japanese on stage reached the audience, but in Brussels the reaction was much weaker, even though the actors’ performances were strong. He suggests that the problem was not the quality of acting, but the context: the key aesthetic element—that Japanese was spoken by non-native speakers—did not become visible to that audience. This difference in response made me think of “context” as another kind of dimension: the same lines and movements can exist in one physical space, yet they are interpreted in very different “cultural spaces.”

The workshop with Mr.\ Leon Koh Yonekawa provided a deeper understanding of these concepts through physical interaction. During the Q\&A session in class, I asked Mr. Yonekawa a question which had been puzzling me regarding the actors' movements. I observed that their positioning on stage seemed random, so I asked: ``If an actor cannot `read the air' (\textit{kuuki wo yomenai} in Japanese), wouldn't the positioning become chaotic and ruin the performance?'' 

His answer was profound and completely changed my perspective. He said that there is no need to ``read the air.'' He explained that wherever an individual stands, that specific spot becomes a valid performance space. It is not about fitting into a pre-existing harmony or following invisible social rules or something. Rather, the act of standing itself defines the space. This randomness is not chaos, but a vital part of the performance loop where each actor asserts their own spatial reality.

This idea of defined space was further explored during a conversation I had with Mr. Yonekawa on our way home. We discussed the mathematical concept of ``spaces of different dimensions.'' As a mathematics student, I proposed that the stage is not just a simple 3-dimensional Euclidean box. In mathematics, we deal with spaces of varying dimensions ($\mathbb{R}^n$), where properties change depending on the dimensional framework.

I suggested that while the actors exist in physical 3D space, the ``In-Between'' narrative creates a higher-dimensional space—a topological structure where distance is measured not in meters, but in understanding and language. Mr. Yonekawa found this mathematical analogy interesting. We discussed how an actor moving on stage is not just translating coordinates in $x, y, z$ axis, but is effectively moving between different dimensional layers. Just as a lower-dimensional object cannot fully comprehend a higher-dimensional one, the characters in the play (humans, robot, alien) struggle to connect because they are essentially operating in different dimensions of existence.

Okada’s column also hints at this dimensional gap. He writes that even when the spoken words are the same, there is always something “extra” in language that meaning alone cannot fully encompass. For me, this “extra” element is like an invisible axis in a higher-dimensional space: it cannot be drawn easily, but it changes how close or far we feel from each other. When the Brussels audience did not sense the special role of non-native Japanese, it was as if they were living in a different coordinate system, where that extra axis was missing.

In conclusion, \textit{The Window of the Spaceship ``In-Between''} is a philosophical inquiry into how we connect with “the other.” Through the articles and the workshop Q\&A, I learned that performance space is defined by presence, not by social atmosphere or ``reading the air.” The Chelfitsch\_note article shows how ``incorrect” Japanese and intercultural casting become tools to build new networks between people, while Okada’s column reveals how fragile this project is when it enters a different cultural dimension. Furthermore, my discussion with Mr.\ Yonekawa confirmed that theatre and mathematics share a common goal: describing the complex structures of our world. Whether through the abstract dimensions of mathematics or the physical and cultural dimensions of the stage, we are all trying to find our coordinates in the vast spaceship of existence.

\newpage

% -----------------------------------------------------------------
% 参考文献部分
% -----------------------------------------------------------------
\begin{center}
    \textbf{References}
\end{center}

\noindent
\vspace{-1em} 

{
\setlength{\parindent}{0cm}
\setlength{\hangindent}{1.27cm}

Chelfitsch\_note. (2023, July 23). New form of intercultural theatre that connects the audience with other---Theatre project with non-native speakers of Japanese [Blog post]. Note.
\url{https://note.com/chelfitsch_note/n/ndc9fb058d4cd}

Okada, T. (2024, September 11). It's tough to handle the language [Column]. Kyoto Experiment. \url{https://kyoto-ex.jp/magazine/2024_etto_etto-1-2/}

Okada, T. (n.d.). The Window of the Spaceship ``In-Between'' [Play script distributed in class].
}

\end{document}