% !!! 编译说明: 请使用 XeLaTeX 编译器

\documentclass[12pt]{article}

% -----------------------------------------------------------------
% 1. 页面设置
% -----------------------------------------------------------------
\usepackage[a4paper, margin=2.5cm]{geometry}

% -----------------------------------------------------------------
% 2. 行距设置
% -----------------------------------------------------------------
\usepackage{setspace}
\doublespacing

% -----------------------------------------------------------------
% 3. 字体设置
% -----------------------------------------------------------------
\usepackage{fontspec}
\setmainfont{Times New Roman}

% -----------------------------------------------------------------
% 4. 段落格式
% -----------------------------------------------------------------
\usepackage{parskip}
\setlength{\parindent}{1.27cm} 
\setlength{\parskip}{0pt}      

% -----------------------------------------------------------------
% 5. 页眉设置
% -----------------------------------------------------------------
\usepackage{fancyhdr}
\pagestyle{fancy}
\fancyhf{} 
\rhead{\thepage} 
\renewcommand{\headrulewidth}{0pt}

% -----------------------------------------------------------------
% 6. 个人信息录入
% -----------------------------------------------------------------
% 注意:右边的引号我帮你改成了 '' (两个单引号),这样才会显示弯引号 ”
\newcommand{\myTitle}{Reflection on One-to-One performance:\\ ``Meeting a Chinese Fan''} 
\newcommand{\myAuthor}{Yi Ran -- 21122200512}
\newcommand{\myAffiliation}{Department of Mathematical Sciences, Ritsumeikan University} 
\newcommand{\myCourse}{35091: Cross-cultural Seminar (G1)}
\newcommand{\myInstructor}{Dr. Shoji}
\newcommand{\myDate}{\today}

% -----------------------------------------------------------------
% 正文开始
% -----------------------------------------------------------------
\begin{document}

% =================================================================
% 封面页 (手动排版,不使用 titlepage 环境以保证页码连续)
% =================================================================
\begin{center}
    \vspace*{3cm} % 下移
    
    \textbf{\Large \myTitle} % 标题
    
    \vspace{2\baselineskip} % 空行
    
    % 信息
    \myAuthor \\
    \myAffiliation \\
    \myCourse \\
    \myInstructor \\
    \myDate
\end{center}

\newpage % 强制换页!这里之后就是第 2 页
% =================================================================

% --- 正文部分 ---

% 正文第一页顶部重复标题
\begin{center}
    \textbf{\myTitle}
\end{center}
\vspace{\baselineskip} 

My performance, \textit{Meeting a Chinese Fan}, was designed to share an essence of Chinese culture not through physical objects, but through a shared atmosphere of calm. Rather than relying on elaborate sets or technology, I intended to use my instructions as tools to guide the audience into a state of quiet introspection. My plan for this ten-to-fifteen-minute session centered on the keyword ``exchange'', aiming to dismantle the traditional barrier between performer and audience. By inviting the audience to swap roles with me—becoming an equal partner rather than a passive observer—I hoped to create a safe environment where we felt free to speak or remain silent, knowing that silence itself can be a profound form of communication in One-to-One encounters.

Many people asked me, ``What inspired this specific approach?'' The primary inspiration was the Chinese fan itself, a deceptively simple object that carries deep cultural associations with summer, cooling wind, and the act of care. Beyond the object, I was drawn to the power of deliberate speech to act as a gentle guide. In the setting of a one-to-one performance, where the physical distance between actors is collapsed, language becomes a bridge. I believed that a soft vocal pace could help the audience transition from their busy daily rhythm into a private, shared space of attention.

On the day of the presentation, I established a small area designated as the ``Fan Meeting Point'', waiting for a participant to step into this created space naturally. Upon the participant's arrival, I greeted him/her with a simple framing of our time together, ``Hello, today you will meet a Chinese fan. We will share wind and memory here.''

We began with a ``Encounter'' to settle our energy, which naturally flowed into an ``Exchange of Wind.'' Here, I used the fan to establish a rhythmic connection before handing the Chinese fan to the participant, inviting him/her to fan me in return. This shift in agency was important, as it physically demonstrated our shared responsibility for the performance. As the wind continued, I asked, ``When you feel wind, what memory or images come to you?'' While they reflected on this question, I fanned the air toward other areas around him/her without actually directing the breeze at him/her, because I wanted to create the illusion that I was continuously fanning him/her, as a way to test whether the participant had entered a state of mental flow. And then, I asked him/her to open  his/her eyes and gave him/her a small piece of paper to draw the image of the wind by my description. Firstly, I traced simple lines on paper to capture the essence of their story, eventually handing the pen over so they could draw my feeling in return.

However, the actual impact of this structure differed significantly from my intentions. Although I aimed to create calm, contrast, and reflection, the feedback from one participant revealed a stark disconnect between my goals and his experience. He offered a critical perspective, noting that the title \textit{Meeting a Chinese Fan} felt like a misleading pun. He spent the session wondering if I had an ``enthusiastic supporter'' rather than focusing on the object itself. This misdirection, combined with my confusing instructions---telling him to ``sit anywhere'' only to move him to a specific spot later---made him feel that the experience was ``pointless'' and untrustworthy.

Actually, this was also part of the performance, because what I said was only a suggestion, just like a whisper of the wind. You could also have said ``no'' to me, or just stayed here and done nothing. But unfortunately, no audience member ever said ``no'' to me, so I had to follow my original plan. Although it might have seemed chaotic, that confusion was exactly what I wanted to express that your thoughts should be as free as the wind, not led by someone else’s direction. As for the title ``fan,'' I think it works whether you understand it as ``a Chinese fan'' (the object) or ``a fan of Chinese culture''. In either case, it doesn’t really affect the core purpose of the performance. Nevertheless, I remain truly grateful for this audience member’s feedback. This experiment showed me that my performance might be too complex. I acknowledge that my guidance may not have been entirely clear, and I recognize that it is a flaw that I need to address.

\end{document}